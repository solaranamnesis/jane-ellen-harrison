\documentclass[a4paper, 11pt, oneside, polutonikogreek, french]{article}
\usepackage[T1]{fontenc}
\usepackage[default]{frcursive}

% Load encoding definitions (after font package)

\usepackage{textalpha}
\usepackage{graphicx}
\setlength{\emergencystretch}{15pt}
\graphicspath{ {./ } }
\usepackage[figurename=]{caption}
\usepackage{float}

\usepackage{listings}
\lstset{basicstyle=\ttfamily}

% Babel package:
\usepackage[french]{babel}

% With XeTeX$\$LuaTeX, load fontspec after babel to use Unicode
% fonts for Latin script and LGR for Greek:
\ifdefined\luatexversion \usepackage{fontspec}\fi
\ifdefined\XeTeXrevision \usepackage{fontspec}\fi

% "Lipsiakos" italic font `cbleipzig`:
\newcommand*{\lishape}{\fontencoding{LGR}\fontfamily{cmr}%
		 \fontshape{li}\selectfont}
\DeclareTextFontCommand{\textli}{\lishape}

\usepackage{booktabs}
\setlength{\emergencystretch}{15pt}
\usepackage{fancyhdr}
\usepackage{microtype}
\begin{document}
\begin{titlepage} % Suppresses headers and footers on the title page
	\centering % Centre everything on the title page
	%\scshape % Use small caps for all text on the title page

	%------------------------------------------------
	%	Title
	%------------------------------------------------
	
	\rule{\textwidth}{1.6pt}\vspace*{-\baselineskip}\vspace*{2pt} % Thick horizontal rule
	\rule{\textwidth}{0.4pt} % Thin horizontal rule
	
	\vspace{1\baselineskip} % Whitespace above the title
	
	{\scshape \LARGE Aegis-Ἀγρηνόν}
	
	\vspace{1\baselineskip} % Whitespace above the title

	\rule{\textwidth}{0.4pt}\vspace*{-\baselineskip}\vspace{3.2pt} % Thin horizontal rule
	\rule{\textwidth}{1.6pt} % Thick horizontal rule
	
	\vspace{1\baselineskip} % Whitespace after the title block
	
	%------------------------------------------------
	%	Subtitle
	%------------------------------------------------

 	\vspace*{1\baselineskip} % Whitespace under the subtitle

	{\scshape \Large Par Jane Ellen Harrison} % Subtitle or further description

 	\vspace*{1\baselineskip} % Whitespace under the subtitle

	%------------------------------------------------
	%	Editor(s)
	%------------------------------------------------
        \vspace*{\fill}

	\vspace{1\baselineskip}

	{\scshape London, 1900 \\Journal of Hellenic Studies, Vol. 20}
		
	\vspace{0.5\baselineskip} % Whitespace after the title block

        \scshape Internet Archive Online Edition% Publication year
	
	{\scshape\small Utilisation non commerciale --- Partage dans les mêmes conditions 4.0 International} % Publisher
\end{titlepage}
\setlength{\parskip}{1mm plus1mm minus1mm}
\clearpage
\paragraph{}
En discutant l'origine de l'omphalos dans le \emph{Journal Hellénique} (\emph{JHS}, 19, 1899, p. 225) j'avais laissé en dehors de mes observations la couverture de tæniæ, le réseau ou filet, dont il est revêtu dans la plupart des représentations ; je n'avais en effet aucune vue précise sur le sens de cet objet, et je le considérais, faute de mieux, comme simplement décoratif. Mais depuis lors, il m'a semblé trouver une théorie plus probable\footnote{Elle a été exposée dans une réunion de la Société des Études Grecques à Cambridge le 5 Mai 1900, et elle est résumée dans le \emph{Journal of Hellenic Studies} (20, p. 44).} ; c'est cette théorie que je me propose de soumettre au jugement des lecteurs du \emph{Bulletin de Correspondance Hellénique}, hommage reconnaissant envers les excavateurs de Delphes et l'École française, en souvenir de l'accueil généreux et de la bienveillante hospitalité dont j'ai si souvent profité.

Il y a beaucoup de variété dans les représentations du réseau, en filet, ou bandelettes, qui recouvre l'omphalos ; parmi les spécimens rassemblés par le Professeur Middleton (\emph{Le Temple d'Apollon à Delphes}, \emph{JHS}, 1888, p. 295-301), on en remarque trois qui représentent indubitablement des bandeaux, entremêlés une fois de rameaux de lauriers ; dans les autres cas, c'est plutôt un filet qui recouvre l'omphalos, mais un filet à dessins divers. A ces représentations tirées de peintures de vases, de types monétaires, etc., l'obligeance de M. Homolle me permet d'adjoindre un exemple d'une valeur singulière, l'omphalos même découvert pendant les fouilles de Delphes (voir ci-dessous la figure). Cet omphalos est entièrement recouvert d'un ornement qui représente évidemment une copie en pierre d'un filet de laine.

Je n'avais encore formulé aucune interprétation précise du réseau de l'omphalos, lorsque j'ai assisté à une communication sur l'égide d'Athéna, faite par le Professeur Ridgeway à Cambridge, aux membres de la branche filiale de la Société Hellénique, communication dont il n'a paru qu'un court résumé (\emph{JHS}, 20, p. 44). En soutenant la thèse que le Gorgoneion était à son origine la tête de chèvre (αἴξ) encore attachée à la peau de l'égide, il insista sur ce fait que la vraie histoire de l'égide a été connue d'Hérodote, et que cette égide n'est autre chose que la robe de peau de chèvre portée encore du temps de cet auteur par les Libyennes (Hérod., 4, 188, 189) et, dans les Lemps anciens, --- fait sans doute ignoré d'Hérodote --- par les Arcadiennes et les Athéniennes.\footnote{Un bon exemple de la coutume de porter la peau de chèvre avec la tête existe chez Élien (\emph{Hist. An}., 1, 23). Il raconte que les poissons appelés \emph{sargoi} avaient un goût marqué pour les chèvres et que, lorsqu'un pêcheur voulait prendre de ces poissons, il se déguisait en chèvre portant la peau de cet animal avec les cornes. « Ἁλιεὺς γὰρ ἀνὴρ αἰγὸς δορᾷ ἑαυτὸν περιαμπέχει σὺν αὐτοῖς τοῖς κέρασι δαρείσης αὐτῆς... ἑλκόμενοι δὲ οἱ σαργοὶ... κηλοῦνται ὑπὸ τῆς δορᾶς. » De plus il cite (16, 34), de Nymphodore, une description de l'île de Sarde, célèbre, dit-il, pour les peaux de chèvres que les habitants portaient comme vêtements : « τὰς γάρ τοι δορὰς τοὺς ἐπιχωρίους ἠπθῆσθαι. » Même, ajoute-t-il, grâce à quelque propriété mystérieuse, φύσει τινὶ ἀπορρήτῳ, ces peaux maintenaient chez celui qui les portait la chaleur en hiver et la fraîcheur en été. Mais le mystère s'explique, quand il raconte qu'on les portait alternativement avec le poil en dedans, et en dehors. Il est intéressant d'apprendre aussi d'Élien (17, 34) que les plus fines toisons des chameaux caspiens servaient comte vêtements aux prêtres (οὔκουν ἐκ τούτων οἱ ἱερεῖς ἐσθῆτας ἀμφιέννυνται).} On demandera peut-être quel rapport il peut y avoir entre la peau de chèvre de l'égide et le filet de l'omphalos. Consultons les lexicographes et les scholiastes, qui nous indiqueront le rapport, peu apparent au premier abord. L'explication d'Hésychius est la plus détaillée, et c'est probablement d'elle que dérivent les autres ; il écrit : « αἰγίς· ὀξεία πνοὴ καὶ ἣν αἱ Λίβυσσαι φέρουσι δοράν· καὶ ἡ ἀπόστιλψις τῶν ὀμμάτων ; » --- et plus loin : « αἰγίς· ὅπλον ἐξ αἰγοῦ. καὶ τὸ ἐκ τῶν στεμμάτων διαπεπλεγμένον δίκτυον. » Sous le mot αἰγιάδες, il donne ὑμένες. Suidas, s. v. Αἰγίδας, dit : « τὸ ἐκ τῶν στεμμάτων πλέγμα καὶ τὸ διὰ στεμμάτων πεπλεγμένον δίκτυον. » Bekker, \emph{Anecd.}, p. 354, s. v. Αἰγίδες : τὸ ἐκ τῶν στεμμάτων πλέγμα καὶ τὸ διάστεμμα τῶν πεπλεγμένων καὶ τὸ Διὸς ὅπλον. Les lexicographes sont donc d'accord pour attribuer au mot \emph{aigis} une double signification, celle d'un manteau ou d'un bouclier de peau de chèvre, et celle d'un réseau de \emph{stemmata} ou bandeaux de laine. Où chercher le lien des deux sens, où trouver l'élément essentiel qui permet de donner à ces deux objets différents une commune appellation ?

Chez les lexicographes, les deux significations se trouvent côte à côte ; ceux-ci, selon leur habitude, ne se sont pas embarrassés de les mettre d'accord. Eustathe au contraire, ne comprenant pas cette double signification, ne laisse pas de nous faire voir ses perplexités. En commentant αἰγίδα θυσανόεσσαν (\emph{ad Il.}, p. 603), après un résumé intelligent et correct de la théorie d'Hérodote, théorie que lui-même accepte comme vraie, il note la seconde signification du mot : \emph{ægis}, dit-il, signifie non seulement αἴγων δέρας, mais encore autre chose, ἄλλο τι ; et il le témoignage d'Ælius Dionysius, écrivain du temps d'Hadrien : « Αἴλιος μὲν γὰρ Διονύσιός φησιν αἰγὶς τὸ ἐκ τῶν στεμμάτων δίκτυον, » et celui de Pausanias (le lexicographe sans doute) : « αἰγὶς... τὸ διὰ τῶν στεμμάτων δίκτυον ; » mais, ajoute-t-il, quel objet peut être ce \emph{filet-égide}, voilà ce qu'ils n'expliquent pas. Lui-même hasarde une conjecture. Il est probable, dit-il, que ces égides sont en laine, comme par exemple lorsqu'on dit στέμματα ξήνασα,\footnote{Eustathe cite l'\emph{Oreste} d'Euripide (v. 12), où Électre parle des fils tissés par la Destinée pour Atrée :\\\hspace*{10mm}ᾦ στέμματα ξήνασ' ἐπέκλωσεν θεὰ\\\hspace*{10mm}ἔριν.\\\hspace*{10mm}Il est clair que le mot στέμματα est ici employé pour désigner non pas des guirlandes, mais simplement des fils de laine.} et les filets (δίκτυα) qu'on en fait ne sont pas exactement des filets de pêche (μὴ αὐτόχρημα δίκτυα γίνεσθαι ἁλιευτικά), mais plutôt ceux dont on se servait plus tard comme bourses (ὁποῖα δὲ τὰ εἰς ζώνην παρὰ τοῖς ὕστερον χρήσιμα). Si, dans le cardage, on laissait des touffes de laine brute, on arriverait justement à produire ces quasi-égides (οἷς ἐπανθοῦν τι χνοῶδες ἐν τῷ γναφεύεσθαι ἀποτελοίη ἂν αἰγιδοειδῆ αὐτά). Inconsciemment, Eustathe touche à la solution ; en tout cas, il met le doigt sur le nœud de la difficulté. On a donné un sens trop limité au mot égide ; ce mot s'applique à n'importe quel objet dérivant de la chèvre, robe en peau, filet de poil, ou bouclier en cuir de chèvre.

Le mot δίκτυον aussi a pris avec le temps un sens trop étroit ; pour un homme, filet ne représente guère qu'un instrument à prendre le poisson ou le gibier ; les femmes cependant se servent d'un filet pour porter leurs emplettes, pour retenir leur chevelure, elles en font même un excellent manteau sous la forme d'un châle de fine laine tricotée. A-t-on lieu de croire que chez les anciens, on s'habillait avec le filet ? Le δίκτυον était-il un ἔνδυμα ? Il est certain qu'il existait un vêtement de cette sorte appelé du nom d'ἀγρηνόν. En parlant des différentes espèces de filet, filet à jeter (δίκτυον), filet tendu (ἄρκος), Pollux (\emph{Onom.}, 4, 116) définit l'ἀγρηνόν : « πλέγμα ἐξ ἐρίων δικτυοειδὲς περὶ πᾶν τὸ σῶμα, ὃ Τειρεσίας ἐπεβάλλετο ἤ τις ἄλλος μάντις ; » puis il décrit les divers vêtements portés sur la scène, en notant la variété des peaux de bête employées à cet usage.\footnote{Je dois mes remercîments au Dr Sandys, qui m'a rappelé la description Eur., \emph{Bacch.}, 111 :\\\hspace*{10mm}στικτῶν τ' ἐνδυτὰ νεβρίδων\\\hspace*{10mm}στέφετε λευκοτρίχων πλοκάμων.\\\hspace*{10mm}Ici les difficultés, bien connues des éditeurs, disparaissent, si l'on suppose qu'un ἀγρηνόν die laine était porté par-dessus la \emph{nébride} (νεβρίς). Je dois ajouter que, pour expliquer de mystérieux στέφειν, en parlant des offrandes de libations (Soph., \emph{Ant.}, 431 : \emph{Élect.}, 52, 548), je n'y voudrais trouver aucune association avec des guirlandes, aucune idée de couronnement (στέφανος), mais une allusion à l'offrande de la laine (στέμματα), sacrifice coutumier pour les divinités des régions souterraines. Dans Eur., \emph{Oreste}, στέμματα doit évidemment avoir la signification, non pas de guirlandes, mais de fils de laine.} Hésychius attribue l'ἀγρηνόν au dieu Dionysos : « ἀγρηνὸν δικτυοειδὲς καὶ ἔνδυμα δὲ ποιόν, » et l'\emph{Etym. magn.} écrit : « ἀγρηνόν· ποίκιλον, ἐρεοῦν, δικτυοειδές. »

Heureusement il existe une représentation de l'ἀγρηνόν servant de vêtement. Un torse du Vatican (Fig. 1), publié par Gerhard, \emph{Antike Bildw.}, Taf. 84, 3, et reproduit par Saglio, \emph{Dictionnaire des Antiquités}, au mot ἀγρηνόν, nous montre un prêtre portant ce vêtement par-dessus le \emph{chiton} et sous la \emph{chlamyde}, qui est attachée par une broche ( ? ) en forme de masque. Gerhard voit dans ce torse un prêtre d'Apollon ; il est peut-être plus plausible encore d'y reconnaître, avec M. Saglio, un prêtre de Dionysos ; en tout cas, c'est sans aucun doute un prêtre habillé en \emph{mantis}. M. Saglio note l'analogie de ce vêtement avec le réseau de l'omphalos.

\begin{figure}[ht]
\centering
\includegraphics[height=0.65\textheight,keepaspectratio]{agrenon-fig1.png}
\end{figure}
\paragraph{}
Le lecteur aura deviné ma conclusion : le δίκτυον de l'omphalos n'était autre chose que son ἀγρηνόν, son vêtement ou son manteau de peau de chèvre, son égide, le filet de poil tricoté ou tissé n'étant qu'une transformation adoucie du rude vêtement de peau.

Il était naturel de vêtir l'omphalos, la pierre de l'ὄμφη, la sainte voix prophétique ; n'était-elle pas en effet un ἔμψυχος λίθος, une haleine vivante, un objet quasi-humain, qu'il fallait chérir et dorloter comme un enfant ? On objectera peut-être qu'il ne reste à Delphes aucune trace d'une telle croyance, que l'omphalos de Delphes (Fig. 2) n'était que le siège vénéré, le trône mantique d'Apollon ; et ce sera vrai pour l'omphalos. Mais il existait à Delphes une autre pierre, celle de Kronos, qu'on arrosait d'huile tous les jours, et enveloppait de laine brute aux jours de fête (Paus., 10, 24, 5). M. J. G. Frazer, dans son commentaire sur Pausanias cite des exemples de rites analogues chez les peuples sauvages ; mais il n'est pas nécessaire de cher des analogies parmi les barbares : dans la Grèce même, la coutume existait ; on habillait une pierre de vêtements fins, on la traitait comme un enfant vivant. J'ai déjà eu occasion (\emph{Delphika}, \emph{JHS}, 1899, p. 240) de parler du λίθος αὐδήεις du poème des \emph{Lithika}, la pierre donnée par Phébus à Hélénus. On traitait cette pierre absolument comme si elle eût été un tout jeune enfant ; on la lavait, on l'habillait de fins vêtements, enfin celui qui consultait l'oracle la berçait dans ses bras, jusqu'à ce qu'elle fît entendre sa voix :
\begin{quotation}
ἀενάῳ δ' ἐνὶ πέτρον ἐχέφρονα πίδακι λούων

φάρεσιν ἐν μαλακοῖσιν ἅτε βρέφος ἀλδήσασκε

καὶ θεὸν ὡς λιπαροῖσιν ἀρεσσάμενος θυέεσσιν.

(\emph{Lith.}, v. 362-6).
\end{quotation}
\paragraph{}
Evidemment cette pierre n'était pas autre chose qu'un omphalos, prononçant des \emph{omphæ}. De plus, dans la description de l'aspect du λίθος αὐδήεις, on trouve un souvenir de l'ἀγρηνόν avec son fin réseau tissé, αἰγίδες ὑμένες.
\begin{quotation}
ἀμφὶ δέ μιν κύκλῳ περὶ τ' ἀμφὶ τε πάντοθεν ἶνες

ἐμφερέες ῥυτίδεσσιν ἐπιγράβδην τανύονται.

(\emph{Lith.}, v. 358-9).
\end{quotation}
\paragraph{}
L'omphalos donc, la pierre qui parle, portait le vêtement du \emph{mantis}. A l'origine, le \emph{mantis} était simplement vêtu de peau de chèvre comme ses concitoyens, puis, bien longtemps après que ceux-ci eurent adopté des tissus différents, le prophète, suivant les habitudes conservatrices du sacerdoce, garda sa peau de chèvre comme robe de culte. Il est possible aussi qu'il dût porter la peau de l'animal sacrifié, afin de se concilier le dieu, en se confondant autant que possible avec la victime. Voilà au moins ce qui semble avoir été le motif du culte d'Oropos (Paus, 1, 34, 5), selon lequel quiconque voulait consulter Amphiaraos sacrifiait un bélier et se couchait ensuite sur la victime en attendant la révélation du rêve : « κριὸν θύσαντες καὶ τὸ δέρμα ὑποστρωσάμενοι καθεύδουσιν ἀναμένοντες δήλωσιν ὀνείρατος. »

Reste à considérer l'étymologie du mot ἀγρηνόν. Au premier abord, on y pourrait voir la même racine que dans le verbe ἀγρεύω, chasser. Remarquons pourtant que, s'il faut croire Pollux et Hésychius, le mot était exclusivement employé pour désigner une robe ou un vêtement, et ne s'appliquait jamais au filet de chasse. Notons aussi qu'Hésychius cite l'autorité d'Ératosthène pour une forme γρῆνυν ou γρῆνον. Serait-il possible --- je ne suggère celle conjecture qu'en passant --- que cette forme soit la même que γρῖνον, citée par Eustathe comme la forme éolienne de ῥῖνος, peau de bête ? En ce cas, l'α ne serait que \emph{prosthétique} et le mot ἀγρηνόν signifierait, comme le mot égide, ce qui est fait d'une peau de bête, et aussi le filet ou vêtement fait de poil. Eustathe parle aussi du mot γριπεύς comme ὁ ῥάπτων ἁλιευτικὰ δίκτυα.

\begin{figure}[ht]
\centering
\includegraphics[height=0.45\textheight,keepaspectratio]{agrenon-fig2.png}
\end{figure}
\paragraph{}
Quelle que soit l'étymologie du mot ἀγρηνόν, il est prouvé que le filet pouvait servir comme vêtement ; on demandera si on employait jamais le vêtement comme filet. Le Pr. Ridgeway a eu l'obligeance de me raconter qu'en Irlande on a connu des moines qui prenaient le poisson avec leur capuchon ; mais je n'ai pas trouvé d'habitude analogue chez les Grecs. Il est cependant permis de croire que le souvenir d'un filet-vêtement hantait l'esprit d'Eschyle lorsqu'il écrivait :
\begin{quotation}
ἄπειρον \emph{ἀμφίβληστρον} ὥσπερ ἰχθύων
    
περιστειχίζω πλοῦτον \emph{εἴματος} κακόν.

(Esch., \emph{Ag.}, 1382).
\end{quotation}
\paragraph{}
Fatalité ironique des choses ! Ce fut son \emph{égide} que Clytemnestre jeta sur son mari !

Disons un mot pour conclure sur les relations du gorgoneion avec l'omphalos. Dans son édition de l'\emph{Ion} d'Euripide (Introduction, p. 46), le Dr Verrall commente les lignes :
\begin{quotation}
ΧΟ.

ἆρ' ὄντως μέσον ὀμφαλὸν

γᾶς Φοίβου κατέχει δόμος ;

\bigskip

ΙΩΝ.

στέμμασί γ' ἐνδυτόν· ἀμφὶ δὲ Γοργόνες.

(Eur., \emph{Ion}, v. 225).
\end{quotation}
\paragraph{}
et fait l'intéressante suggestion que les Gorgones dont parle le poète étaient de grossières représentations placées de chaque côté de l'omphalos, formes vagues, connues quelquefois sous le nom de \emph{Moiræ}, les Destinées, mais que les prêtres de Zeus disaient être ses aigles. Sans discuter cette explication, je voudrais toutefois la modifier. Il a été prouvé par le Pr. Ridgeway (\emph{art. cité}) que le \emph{Gorgoneion} sur l'égide n'est pas autre chose que la tête de la bête attachée à sa peau, portée sur la poitrine en guise d'ἀποτροπαῖον, comme on le voit encore de nos jours chez les peuples sauvages. Supposons que l'omphalos ait porté à l'origine un vêtement de peau de chèvre ; la tête y restera attachée ; voilà déjà un \emph{gorgoneion}. L'omphalos étant sphérique, il est possible qu'on y ait employé deux peaux pour le bien draper ; en ce cas, la seconde tête nous fournira le deuxième \emph{gorgoneion}. Plus tard, lorsqu'on eut remplacé le vêtement de peau par un filet en laine, les deux \emph{gorgoneia} ont bien pu y rester comme broches ou masques, gardant toujours leur valeur prophylactique ou fatale. En dernier lieu, quand l'omphalos passa avec l'oracle à Zeus et à Apollon, on les aurait remplacés par les aigles de Zeus. L'expression στέμμασι γ' ἐνδυτόν s'accorde très bien avec la notion du vêtement ἀγρηνόν, et les mots ἀμφὶ δὲ Γοργόνες s'appliqueraient à la position des masques. Tout cela, il est vrai, n'est qu'hypothèse pure ; toutefois on doit noter que les relations entre les aigles et l'omphalos sont des plus vagues et des plus variables. Dans le relief de Sparte, ces oiseaux sont posés sur un piédestal, un de chaque côté de l'omphalos ; sur les statères de Cyzique, ils y sont accrochés d'une façon maladroite ; dans la majorité des représentations de l'omphalos les aigles ne paraissent pas du tout. Aussi leur présence n'explique-t-elle aucunement la phrase d'Ion, ἀμφὶ δὲ Γοργόνες, \emph{des deux côtés}. Au contraire, notre conjecture est en quelque mesure confirmée par la glose d'Hésychius\footnote{Cette glose a été notée premier lieu, je crois, par Wieseler (\emph{Intorno all' omphalo delfico}, \emph{Annali}, 1857, p. 1781, à propos de l'extrait des lexicographes sur le mot αἰγίδας. Il en conclut que les mots στέμματα et Γοργόνες sont pour ainsi dire en apposition, et il traduit στέμματα \emph{fascie di lana}, Γόργονες, \emph{una rete composta di fascie di lana}, interpretation, selon moi, impossible à maintenir.} : Γοργόνες· αἰγίδες, οἱ δὲ τὰ ἐπὶ τῶν αἰγίδων πρόσωπα. Il parle ici des Gorgones et des αἰγίδες comme étant identiques, ce qui n'offrirait aucune difficulté, si la Gorgone n'était que la tête de chèvre attachée à la peau.
\end{document}
